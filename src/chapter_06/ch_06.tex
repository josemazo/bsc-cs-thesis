\chapter{Conclusiones y trabajo futuro} \label{chap:6}

\vspace*{5mm}

\section{Conclusiones personales} \label{sec:6.1}

¿Qué puedo decir sobre el aprendizaje automático al final de esta pequeña, pero intensa aventura? Me encanta. Sinceramente, el detonante de esto ha sido el resultado del caso práctico: ¿predecir el \emph{comportamiento humano} con cierto grado de acierto? Hace varios años, esto para mí solamente era posible en los libros de \emph{Isaac Asimov} o \emph{Philip K. Dick}. Pero claro está, este resultado práctico no hubiera sido posible sin un estudio del contenido teórico de machine learning, del que sólo he rasgado la superficie.

Pero además, durante este viaje he podido aprender algunas cosas más, quizás no tan valiosas como el \emph{\textbf{conocimiento} de machine learning adquirido y \textbf{cómo usarlo}}, pero no por ello menos importantes:

\begin{itemize}
\item[\textbullet] Leer \emph{textos científicos} de manera algo más crítica.

\item[\textbullet] Utilizar \hologo{LaTeX} para escribir documentación.

\item[\textbullet] Aprender una versión más \emph{científica} de Python.

\item[\textbullet] Utilizar IPython Notebook de manera eficiente.

\item[\textbullet] Utilizar Docker para casos reales.

\item[\textbullet] Resolver problemas que necesitan una gran cantidad de recursos computacionales.
\end{itemize}

En definitiva, aunque pequeño, este proyecto ha sido muy grande para mí, y aunque a veces ha sido algo duro, he disfrutado de él de principio a fin.

\section{Trabajo futuro} \label{sec:6.2}

Pero esto no queda aquí. Como comentaba, tan sólo he rasgado la superficie de lo que ofrece el aprendizaje automático, y deseo continuar estudiándolo. Para ello, tengo que reforzar y ampliar los conocimientos matemáticos adquiridos durante los años en la universidad.

\emph{Estadística}, \emph{álgebra lineal} y \emph{cálculo} conforman una base necesaria antes de empezar con la teoría general que subyace en todo algoritmo de machine learning, como por ejemplo los \emph{principios del aprendizaje computacional}. Mientras se estudia esa teoría común, no sería mala idea alternar de vez en cuando con el estudio de distintos algoritmos, desde el artículo original hasta las modificaciones posteriores más relevantes, y además ponerlos en práctica con algún que otro caso práctico, como por ejemplo, con el potente dataset \emph{nyc\_taxi\_2013}.

A este datset aún se le puede sacar mucho juego. Por ejemplo, se podría unir a la información que tenga disponible la \emph{MTA}, siglas de \emph{Metropolitan Transportation Authority} o \emph{Autoridad de Transporte Metropolitano} en español, sobre los transportes en metro y autobús para detectar los patrones de movimiento de las personas en la ciudad de Nueva York. Así se podría mejorar el flujo de estas, sobre todo en casos de emergencia, donde la rapidez y el orden son vitales. Además, empresas como \emph{Uber}\footnote{Empresa que pone en contacto a pasajeros con conductores particulares para ofrecer un servicio de transporte: \url{https://www.uber.com/}} o \emph{Lyft}\footnote{Servicio que, al igual que Uber, ofrece transporte sobre vehículos de particulares: \\ \url{https://www.lyft.com/}}, o incluso la propia Comisión de Taxis y Limusinas, podrían utilizar estos datos para proporcionar mejores servicios a la hora de compartir transporte, no sólo ofreciendo un medio más barato, sino reduciendo el impacto de la huella de carbono en el medio ambiente.

Cuando \emph{nyc\_taxi\_2013} se quede \emph{pequeño}, existen lugares donde encontrar más datasets y también casos prácticos para aplicar lo aprendido en la parte teórica, como por ejemplo Kaggle\footnote{\url{https://www.kaggle.com/}}. Esta web ofrece, entre otros servicios, retos de diversa magnitud sobre machine learning donde cualquiera puede participar, retos que ponen a disposición de los participantes datos de bastante calidad generados por grupos de investigación, tanto de universidades como de empresas privadas.

Estos casos prácticos de Kaggle son un medio ideal para usar BMO, ya que esta herramienta se podrá ir ajustando a las necesidades de machine learning del momento. Quizá pueda evolucionar a un entorno en la nube donde cualquiera pueda tener, fácil y rápidamente, un lugar donde poner en práctica sus conocimientos de machine learning, o incluso disponer de las herramientas necesarias para la generación, de manera muy simple, de visualizaciones de datos interactivas.

Un largo camino, de varios años, que aunque dará múltiples quebraderos de cabeza, estoy seguro que además de bonito será una gozada.

\section{El posible futuro del machine learning} \label{sec:6.3}

Como las \emph{Ciencias de la Computación} son todavía muy jóvenes, no es de extrañar que se pueda encontrar falta de consenso en algunas descripciones teóricas de machine learning. Esta leve fragilidad llama a la puerta de los oportunistas, los cuáles ponen de moda y deforman términos ya creados, como \emph{Big Data} y \emph{Data Science}. Por supuesto, no estoy en contra de estos términos, pero si de su mal uso para fines comerciales, tanto a nivel de empresas, como de investigación (qué horror), e incluso personal.

Tengo la pequeña esperanza de que el término \emph{Machine Learning} no se desvirtúe, y que su uso sea reconocido en las tantas aplicaciones que utilizarán el aprendizaje automático como la potente herramienta que es. Por ejemplo, en \cite{kantardzic2005next} se listan una serie de artículos donde el machine learning se utilizará en futuras aplicaciones, teniendo éste un rol clave.

Estas aplicaciones pertenecen a ámbitos tales como ingeniería, biotecnología, medicina y hasta humanidades. Energía sin desechos o con independencia total de combustibles fósiles, cepas de cereal que apenas necesitan riego, medicamentos más eficientes pero sobre todo más baratos o detectar corrupción en gobiernos y empresas, son una simple muestra de los problemas que el aprendizaje automático resolverá en estos ámbitos. Pero el resultado común que buscan todas estas aplicaciones consiste en mejorar la calidad de vida actual, y aún más destacable, llevar esa calidad de vida a aquella parte de la población mundial que no dispone ni un ápice de ella.

\emph{¿No es quizás este el mejor pretexto para empezar a estudiar machine learning?}
